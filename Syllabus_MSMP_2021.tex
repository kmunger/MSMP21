\vspace{-.1in}\documentclass[11pt]{article}

% == margins
\addtolength{\hoffset}{-0.75in} \addtolength{\voffset}{-0.75in}
\addtolength{\textwidth}{1.5in} \addtolength{\textheight}{1.5in}
\renewcommand{\baselinestretch}{1}
%% === hyperref options ===
\usepackage{color}
%\usepackage[bookmarks=true, bookmarksopen=true, linkcolor=webred, urlcolor=blue]{hyperref}
\usepackage{chngpage}
\usepackage{lscape}
\usepackage{rotating}
\usepackage{multirow}
\usepackage{arydshln}
\usepackage{natbib}
\usepackage{bibentry}
\usepackage[hyphens]{url}
\usepackage{hyperref}
\nobibliography*

\setlength{\parskip}{.6em}

%% === document starts here

\title{\bf PLSC 597: Media, Social Media \& Politics}
\author{{\bf \Large Fall 2021} \\ \\ Wednesday 2:00--5:00}
\date{Graduate Seminar\\ Penn State University}

\begin{document}

\maketitle

\section*{Instructor}

\noindent Professor: Kevin Munger\\
 \url{kmm7999@psu.edu}\\
Office hours: Schedule by email (usually Thursdays 1-3pm )
%\medskip

\section*{Course Overview}

This seminar covers recent and classic empirical research on the relationship between ``the media'' (broadly understood) and politics. The modern study of mass media influence originated in the 1940s and spans several social science disciplines. As we will see, the paradigms developed in the early years of that research program continue to influence scholars today --- as well as to be debated and critiqued. Some of the canonical questions we will explore include the power of media messages to persuade; the extent to which media diets are ideologically slanted; and the role of new information technologies and social media on societal pathologies such as mass polarization.

It would be impossible to adequately cover all aspects of media research even in a comprehensive survey course. As such, this seminar will focus on relatively recent work that is quantitative in nature (although not exclusively so), but we will also strive to remain grounded in foundational works. Some important pieces are left off under the assumption that Political Science students have been exposed to them in other courses.



\section*{Prerequisites}

Many of the readings will be difficult for those who have not taken at least one semester of graduate statistics. 

\section*{Course Components and Grading}

\begin{itemize}
	
\item {\bf Precis:} You will be expected to complete the readings and submit a 350-500 word reading response paper by midnight on Tuesday before each session. This is a useful commitment device for everyone involved, and helps me understand what people are interested in. (20\%)	
	
\item {\bf Participation:} Given the above, please be prepared to contribute meaningfully to the discussion. Active participation is critical to the seminar format! (10\%)

\item {\bf Literature review:} You will submit a roughly 10-page review of literature on a specific topic. It should critically summarize the state of the empirical research in a specific area and could potentially be useful in the development of your final paper. (20\%) %This is intended to aid students in the development of their JP. Literature reviews should be 5 to 8 pages long and critically summarize the state of the empirical research in a specific area. This includes area of consensus, disagreement, and gaps or blind spots in the literature. (10\%) \\
\item {\bf Research paper:} Your 25-page final paper may involve original data collection, a replication and extension of previous work, or analysis of existing data. All topics must be cleared with me in advance. Papers should follow APSA style guidelines. (50\%) %. Due: {\bf May 8}.
\end{itemize}

%\clearpage
\section*{Books}



%\clearpage
%\section*{Recommended Books}

\noindent We will read multiple chapters of many books for this course. If you're interested in this topic, these are worth purchasing, but I'll make scans of all of the readings available on my website.



\section*{Required Syllabi Statements}

ACADEMIC INTEGRITY STATEMENT
Academic integrity is the pursuit of scholarly activity in an open, honest and responsible manner. Academic integrity is a basic guiding principle for all academic activity at The Pennsylvania State University, and all members of the University community are expected to act in accordance with this principle. Consistent with this expectation, the University’s Code of Conduct states that all students should act with personal integrity, respect other students’ dignity, rights and property, and help create and maintain an environment in which all can succeed through the fruits of their efforts.
Academic integrity includes a commitment by all members of the University community not to engage in or tolerate acts of falsification, misrepresentation or deception. Such acts of dishonesty violate the fundamental ethical principles of the University community and compromise the worth of work completed by others.

DISABILITY ACCOMMODATION STATEMENT
Penn State welcomes students with disabilities into the University’s educational programs. Every Penn State campus has an office for students with disabilities. Student Disability Resources (SDR) website provides contact information for every Penn State campus (http://equity.psu.edu/sdr/disability-coordinator). For further information, please visit the Student Disability Resources website (http://equity.psu.edu/sdr/).
In order to receive consideration for reasonable accommodations, you must contact the appropriate disability services office at the campus where you are officially enrolled, participate in an intake interview, and provide documentation: See documentation guidelines at (http://equity.psu.edu/sdr/guidelines). If the documentation supports your request for reasonable accommodations, your campus disability services office will provide you with an accommodation letter. Please share this letter with your instructors and discuss the accommodations with them as early as possible. You must follow this process for every semester that you request accommodations.

COUNSELING AND PSYCHOLOGICAL SERVICES STATEMENT
Many students at Penn State face personal challenges or have psychological needs that may interfere with their academic progress, social development, or emotional wellbeing. The university offers a variety of confidential services to help you through difficult times, including individual and group counseling, crisis intervention, consultations, online chats, and mental health screenings. These services are provided by staff who welcome all students and embrace a philosophy respectful of clients’ cultural and religious backgrounds, and sensitive to differences in race, ability, gender identity and sexual orientation.
Counseling and Psychological Services at University Park  (CAPS)
(http://studentaffairs.psu.edu/counseling/): 814-863-0395
Counseling and Psychological Services at Commonwealth Campuses
(http://senate.psu.edu/faculty/counseling-services-at-commonwealth-campuses/)
Penn State Crisis Line (24 hours/7 days/week): 877-229-6400
Crisis Text Line (24 hours/7 days/week): Text LIONS to 741741

EDUCATIONAL EQUITY/REPORT BIAS STATEMENT 
Penn State takes great pride to foster a diverse and inclusive environment for students, faculty, and staff. Consistent with University Policy AD29, students who believe they have experienced or observed a hate crime, an act of intolerance, discrimination, or harassment that occurs at Penn State are urged to report these incidents as outlined on the University’s Report Bias webpage (http://equity.psu.edu/reportbias/)



\clearpage
\section*{Schedule}




\subsection*{August 25: Intro, Preliminaries}

%Excerpt from Lazarsfeld, Berelson, and Gaudet, \emph{The People's Choice} (1944). Available at \href{https://canvas.harvard.edu/files/2618993/download?download_frd=1&verifier=LBnQ37UoEjl4EiX5isP4CFPWkPQN8MDw0ofirLV9}{\texttt{this link}}.

\noindent Klapper, Joseph T. 1957. \href{https://github.com/kmunger/MSMP21/blob/main/klapper.pdf}{``What We Know About the Effects of Mass Communication: The Brink of Hope.''}


\noindent James Beniger, 1993. \href{https://github.com/kmunger/MSMP21/blob/main/beniger.pdf}{Communication--Embrace the Subject, Not the Field.}

\subsection*{September 1: The Medium is the Message}


Karl Deutsch, \textit{The Nerves of Government} \href{https://github.com/kmunger/MSMP21/blob/main/1.pdf}{ch 1-2},

\href{https://github.com/kmunger/MSMP21/blob/main/5.pdf}{5-8.}

\noindent Marshall McLuhan, \href{https://designopendata.files.wordpress.com/2014/05/understanding-media-mcluhan.pdf}{ \textit{Understanding Media} ch 1-3}

\noindent \href{https://www.jstor.org/stable/825125?seq=1#metadata_info_tab_contents}{Mcluhan on Deutsch}

\noindent Munger, Kevin and Andy Guess and Eszter Hargittai, 2021. \href{https://journalqd.org/article/view/2713/1825}{Quantitative Description of Digital Media: A Modest Proposal to Disrupt Academic Publishing}


\subsection*{September 8: Time}


\noindent Bennett, W. Lance and Shanto Iyengar. 2008. \href{https://github.com/kmunger/MSMP21/blob/main/bennett.pdf}{``A New Era of Minimal Effects? The Changing Foundations of Political Communication.'' }


\noindent Kevin Munger. 2019. \href{https://journals.sagepub.com/doi/full/10.1177/2056305119859294}{The Limited Value of Non-Replicable Field Experiments in Contexts With Low Temporal Validity}


\noindent Kevin Munger, \href{https://osf.io/ca5wz/}{Temporal Validity}

\noindent David Karpf. 2012. \href{https://github.com/kmunger/MSMP21/blob/main/karpf2012.pdf}{Social Science Research Methods In Internet Time }

\noindent David Karpf, 2019. \href{https://github.com/kmunger/MSMP21/blob/main/karpf2020.pdf}{Two provocations for the study of digital politics in time}



\subsection*{September 15: Recent History}


\href{https://github.com/kmunger/MSMP21/blob/main/tech.pdf}{Neil Postman, \textit{Technopoly} ch 1-4.}

\noindent \href{https://github.com/kmunger/MSMP21/blob/main/control.pdf}{James Beniger, \textit{The Control Revolution} ch 1, 8.}

\noindent \href{https://github.com/kmunger/MSMP21/blob/main/lotz.pdf}{Amanda Lotz, \textit{The Television Will Be Revolutionized}, ch attached.}


%\noindent Bartels, Larry M. 1993. ``Messages Received: The Political Impact of Media Exposure.'' \emph{American Political Science Review} 87(2): 267--285.


%\noindent Jerit, Jennifer and Jason Barabas. 2011. ``Exposure Measures and Content Analysis in Media Effects Studies.'' In \emph{The Oxford Handbook of American Public Opinion and the Media}, eds. Robert Y. Shapiro and Lawrence R. Jacobs. Oxford: Oxford University Press, 139--155.

%\noindent Vavreck, Lynn and Shanto Iyengar. 2011. ``The Future of Political Communication Research: Online Panels and Experimentation.'' In \emph{The Oxford Handbook of American Public Opinion and the Media}, eds. Robert Y. Shapiro and Lawrence R. Jacobs. Oxford: Oxford University Press, 156--168.

%\noindent Green, Donald P., Brian R. Calfano and Peter M. Aronow. 2014. ``Field Experimental Designs for the Study of Media Effects.'' \emph{Political Communication} 31(1): 168--180.

%\noindent Gerber, Alan and Green, Donald. 1999. ``Misperceptions About Perceptual Bias.'' \emph{Annual Review of Political Science} 2(1999): 189--210.

%\noindent Bullock, John G. 2011. ``Elite Influence on Public Opinion in an Informed Electorate.'' \emph{American Political Science Review} 105: 496--515. 



%\noindent Guess, Andrew and Alexander Coppock. 2018. ``Does Counter-Attitudinal Information Cause Backlash? Results from Three Large Survey Experiments,'' \emph{British Journal of Political Science}.

\subsection*{September 22: Positivism, Media Effects }


Hovland, Carl I., Arthur A. Lumsdaine, and Fred D. Sheffield. 1949. \emph{Experiments on Mass Communication.} Princeton: Princeton University Press. \href{https://github.com/kmunger/MSMP21/blob/main/hov1.pdf}{Introduction,} \href{https://github.com/kmunger/MSMP21/blob/main/hov2.pdf}{Chapter 2. }


\noindent Asimovic et al. 2021. \href{https://www.pnas.org/content/118/25/e2022819118}{Testing the effects of Facebook usage in an ethnically polarized setting}



\noindent Sides, Vavreck, Warshaw, Working paper. \href{http://chriswarshaw.com/papers/advertising.pdf}{The Effect of Television Advertising in United States Elections}.

\noindent Coppock, Hill, Vavreck, 2020. \href{https://advances.sciencemag.org/content/6/36/eabc4046?intcmp=trendmd-adv}{The small effects of political advertising are small regardless of context, message, sender, or receiver: Evidence from 59 real-time randomized experiments}.
	



\subsection*{September 29: No Class, APSA}

\subsection*{October 6: Polarization and Civility}


Ladd, Jonathan. 2005. {\it Why Americans Hate the Media and Why It Matters}. ch TBA



\noindent James Fallows, 1996. \href{https://www.theatlantic.com/magazine/archive/1996/02/why-americans-hate-the-media/305060/}{``Why Americans Hate the Media"}


\noindent Prior, Markus. 2013. ``Media and Political Polarization.'' \emph{Annual Review of Political Science} 16:101--127.

\noindent Westwood, Sean, Shanto Iyengar, Yphtach Lelkes, Matthew Levendusky, and Neil Malhotra. 2018. ``The Origins and Consequences of Affective Polarization in the United States.'' \emph{Annual Review of Political Science}, forthcoming.



%\noindent Munger, Kevin. 2017. ``Experimentally Reducing Partisan Incivility on Twitter.'' Working paper available here: \url{http://kmunger.github.io/pdfs/jmp.pdf}


%\subsection*{March 19: Spring Break (no class)}



% MORE

%\emph{Lit reviews due in class.}

\subsection*{October 13: Media Consumption and Selective Exposure}



Prior, Markus. \textit{Post-Broadcast Democracy}, ch TBA


\noindent Sood, Gaurav and Yphtach Lelkes. 2018. ``Don't Expose Yourself: Discretionary Exposure to Political Information.'' \emph{Oxford Research Encyclopedia of Politics}. Available here: \url{http://gsood.com/research/papers/selexp.pdf}

\noindent Guess, Andrew, Benjamin Lyons, Brendan Nyhan, and Jason Reifler. 2018. ``Avoiding the Echo Chamber About Echo Chambers: Why Selective Exposure To Like-Minded Political News Is Less Prevalent Than You Think.'' The Knight Foundation.


\noindent Buntain, Cody et al, 2021. \href{https://dl.acm.org/doi/10.1145/3449085}{YouTube Recommendations and Effects on Sharing Across Online Social Platforms}.
	


% MORE? echo chamber about echo chambers

%\noindent Guess, Andrew, Brendan Nyhan, and Jason Reifler. 2017. ``Selective Exposure to Misinformation: Evidence from the consumption of fake news during the 2016 U.S. presidential campaign.'' Working Paper.

\subsection*{October 20: Digital Prejudice }

Paluck, Elizabeth Levy. 2009. ``Reducing intergroup prejudice and conflict using the media: A field experiment in Rwanda.'' \emph{Journal of Personality and Social Psychology} 96(3): 574--587.

\noindent Matias, J. Nathan. 2019. \href{https://www.pnas.org/content/pnas/116/20/9785.full.pdf}{Preventing harassment and increasing group participation through social norms in 2,190 online	science discussions}.

\noindent Lajevardi, Nazita, 2021. \href{https://www.journals.uchicago.edu/doi/full/10.1086/711300?casa_token=wITxV2WJeaQAAAAA:1wco3QZFwUnMq6EOzYON4LXw2XZUU-By0eeehfQq8QuDzgcUuQW93THNT8o7USkGF81ypdGHcHA}{The media matters: Muslim american portrayals and the effects on mass attitudes}

\noindent Siegel, Alexandra et al. 2021. \href{https://alexandra-siegel.com/wp-content/uploads/2019/08/qjps_election_hatespeech_RR.pdf}{Trumping Hate on Twitter? Online Hate Speech in the 2016 US Election Campaign and its Aftermath}






\subsection*{October 27: Is Social Media Driving Us Apart?}

\noindent Settle, Jaime. 2018. {\it Frenemies} chapters 1-4.

\noindent Boxell, Levi, Matthew Gentzkow, and Jesse M. Shapiro. 2017. ``Greater Internet use is not associated with faster growth in political polarization among US demographic groups.''
\emph{Proceedings of the National Academy of Sciences} 114(40): 10612--10617. (skim)


\noindent Bail et al. 2018. ``Exposure to opposing views on social media can increase political polarization.'' \emph{Proceedings of the National Academy of Sciences} 115(37): 9216--9221.

\noindent Chris Bail, 2021. \textit{The Social Media Prism} ch TBA



%\noindent Bond et al. 2012. ``A 61-million-person experiment in social influence and political mobilization,''  \emph{Nature}.
%
%\noindent Robert M. Bond, Jaime E. Settle, Christopher J. Fariss, Jason J. Jones, and James H. Fowler. 2017. ``Social Endorsement Cues and Political Participation.'' \emph{Political Communication} 34(2): 261--281.
%
%\noindent Jones JJ, Bond RM, Bakshy E, Eckles D, Fowler JH. 2017. `` Social influence and political mobilization: Further evidence from a randomized experiment in the 2012 U.S. presidential election.’’  \emph{PLOS ONE} 12(4): e0173851.
%
%\noindent Tufekci, Zeynep and Christopher Wilson. 2012. ``Social Media and the Decision to Participate in Political Protest: Observations From Tahrir Square.'' \emph{Journal of Communication} 62(2): 363--379.
%
%\noindent Settle, J. E., Bond, R. M., Coviello, L., Fariss, C. J., Fowler, J. H., and Jones, J. J. 2016. ``From posting to voting: The effects of political competition on online political engagement.'' \emph{Political Science Research and Methods} 4(2): 361--378.


\subsection*{November 3: Quantitative Description}






Guess, Andrew, Kevin Aslett, Joshua Tucker, Rich Bonneau and Jonathan Nagler. 2021. \href{https://journalqd.org/article/view/2586}{Cracking Open the News Feed:	Exploring What U.S. Facebook Users See and Share with Large-Scale Platform Data}

\noindent Guess, Andrew, Brendan Nyhan, and Jason Reifler. 2020. \href{https://www.nature.com/articles/s41562-020-0833-x?proof=t}{Exposure to untrustworthy websites in the 2016 US election}.


\noindent Munger, Kevin and Jospeh Phillips, 2020. \href{https://journals.sagepub.com/doi/pdf/10.1177/1940161220964767?casa_token=pNSl7u_rBX0AAAAA:mlH30Wrj5NEIp9nMD459qqBLaUCBlFjwjCwMSGSmd60OIeuomF4XxXH5eBfzk6A3rcmrDE3I9nmK}{Right-Wing YouTube: A Supply and Demand Perspective
}

\noindent Hosseinmardi et al, 2021.  \href{https://www.pnas.org/content/pnas/118/32/e2101967118.full.pdf?casa_token=vI6yIQS3_M8AAAAA:HCt5StQUZ6kzh4X37RqDc0S7qJibnx9wya3wLUZVHjP-Fij7eNUKP1Wb1kpTPFG4k7yvUoRlpdBMEQ}{Examining the consumption of radical content
	on YouTube
}




%\noindent Benkler et al. 2017. ``Partisanship, Propaganda, and Disinformation: Online Media and the 2016 U.S. Presidential Election.'' Link: \url{https://cyber.harvard.edu/publications/2017/08/mediacloud} (skim executive summary and takeaways)


\subsection*{November 10: Media Economics}

%\bibentry{King13}
%
%\noindent \bibentry{King14}

\noindent Hamilton, James. 2004. {\it All the News that's Fit to Sell: How the market transforms information into news}. Princeton: Princeton University Press. Chapters 1-3, 7

\noindent Hindman, Matthew. 2018. \textit{The Internet Trap: How the Digital Economy Builds Monopolies and Undermines Democracy} Chapters TBA.


\noindent Munger, Kevin, 2020. \href{https://www.tandfonline.com/doi/pdf/10.1080/10584609.2019.1687626?casa_token=VSjjEXja4EMAAAAA:Q47ASc0lu3itcu7FAjqifqtnrFQvGFvwCJvG9mE1-g7SDLjLNheR6ZtkURoLvRyW7LRFwxmFR0Di}{All the News that's Fit to Click: The Economics of Clickbait Media.}



%\subsection*{April 30: Discussion of Research Papers + TBD}


\subsection*{November 17: Emerging Platforms}


Ventura, Tiago et al, 2021. \href{https://journalqd.org/article/view/2573/1820}{Connective Effervescence and Streaming Chat During Political Debates}

\noindent TikTok

\noindent ETC



\subsection*{November 24: No Class, Thanksgiving}

\subsection*{December 1:  Protests}

Pablo Gerbaudo, \textit{The Digital Party} ch 2,4,8-10

Zeynep Tufecki, \textit{Twitter and Tear Gas} ch TBA

\noindent Shugars et al, 2021. \href{https://journalqd.org/article/view/2570}{Pandemics, Protests, and Publics: Demographic Activity and Engagement on Twitter in 2020}




\subsection*{December 8: Research Presentation}

Academics, and grad students in particular, tend to avoid presenting work before it is extremely polished. This has some advantages, but we neglect to develop our skills at both presenting and discussing work in progress. 

This week, each person will give a short presentation based on their working paper, with another person serving as discussant.

\subsection*{December 15: Final Papers due to my inbox by 5pm}




\bibliographystyle{plain}
\nobibliography{syllabus}

\end{document}
